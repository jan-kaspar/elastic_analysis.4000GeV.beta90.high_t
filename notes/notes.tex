\input article
\input macros

\def\baseDir{/afs/cern.ch/work/j/jkaspar/analyses/elastic/4000GeV,beta90,high_t}
\def\release{}

%----------------------------------------------------------------------------------------------------

\centerline{\SetFontSizesXX Elastic analysis, $\sqrt s = 8\un{TeV}$, $\be^* = 90\un{m}$, full $|t|$ range}
\vskip2mm
\centerline{version: {\it \number\day. \number\month. \number\year}}

%----------------------------------------------------------------------------------------------------

\section{Introduction}

\> extension of the NPB analysis, with no $|t|$ restriction

%----------------------------------------------------------------------------------------------------

\section{Datasets}

\> DS4 = fill from 12-13 July 2012 (fill number 2836), RPs at 9.5sigma, CMS trigger; runs:
\>> 8369: bunches 648, 2990
\>> 8371: bunches 648, 2990
\>> 8372: bunches 648, 2990, 26


%----------------------------------------------------------------------------------------------------
\section{Data-taking conditions conditions}

\> only indicative numbers here, fore more precise (time-dependent) estimates see the ``Resolution''
section below

\> beam divergence in y: directly observable 2.3 urad

\> beam divergence in x
\>> $\si(vertex) = 168\un{\mu m}$
\>> $\si(vertex) = \si(beam size) / \sqrt 2$, $\si(\hbox{1-arm beam divergence}) = \si(beam size) / \be^*$, hence 
	si(1-arm beam divergence) = 2.4 urad


%----------------------------------------------------------------------------------------------------
\section{Ntuples}

\> Currently used ntuples:

\>> /castor/cern.ch/totem/offline/Reco/2012/Physics/July/v\_3.11


\> Alternative -- common CMS-TOTEM:

\>>	/castor/cern.ch/totem/offline/CMSTOTEM/MergedNtuples/HighBeta/198902-8369\_8371-V00-02-00/RomanPots/ \\
	/castor/cern.ch/totem/offline/CMSTOTEM/MergedNtuples/HighBeta/198903-8372-V00-02-00/RomanPots/"

%----------------------------------------------------------------------------------------------------
\section{Hit distributions}

\> as in NPB

%----------------------------------------------------------------------------------------------------
\section{Alignment}

\> as in NPB

%----------------------------------------------------------------------------------------------------
\section{Optics}

\> as in NPB

%----------------------------------------------------------------------------------------------------
\section{Reconstruction formulae}

\> as in NPB

%----------------------------------------------------------------------------------------------------
\section{Resolution}

\> \plot{resolutions/resolutions_vs_time_per_bunch.pdf} : time dependence of resolution-related quantities, per bunch


%----------------------------------------------------------------------------------------------------
\section{Cuts/elastic tagging}

\> as in NPB
\> cuts 1, 2, 7, 5, 6; not 3 and 4

\> All cuts applied at $5\un{\si}$ level
\>> the ``sigma'' corresponds to the RMS of the distribution, may not include non-Gaussian tails

\> \plot{background,cut_efficiency/t_distributions_n_si.pdf} : $|t|$-distribution with cuts applied at different $n_\si$ levels
\>> few per-mille difference wrt.~the default cut at $5\un{\sigma}$
\>> at higher $|t|$, the relative change is larger, but it corresponds to difference of 1 or 2 events (dashed line represents 1-event difference)


%----------------------------------------------------------------------------------------------------
\section{Background}

\> expectable background
\>> beam halo + ...
\>> DPE, 2x SD: the two protons uncorrelated (for DPE most likely good approximation), thus collinearity cuts
shall yield strong suppression


\> \plot{background,cut_efficiency/cut_distributions.pdf}: distributions of cut discriminators under various cut combinations

\> \plot{background,cut_efficiency/cut_dist_antidgn_cmp.pdf}: one cut is released and its discriminator distributions is plotted, for diagonal and anti-diagonal configurations

\> \plot{background,cut_efficiency/t_distributions_cuts.pdf}: t-distributions under different cut combinations

\> \plot{background,cut_efficiency/t_dist_antidgn_cmp.pdf}: t-distributions, with all cuts but from different diagonal and anti-diagonal configurations


%----------------------------------------------------------------------------------------------------
\section{Binning}

\> \plot{binning/bin_size_vs_t.pdf}
\>> at low $|t|$: bin size = a multiple of smearing sigma
\>> at medium $|t|$: bin size to give a fixed statistical uncertainty
\>> at high $|t|$: bin size not larger than $0.1\un{GeV^2}$
\>> no events beyond $|t| \approx 1.9\un{GeV^2}$

%----------------------------------------------------------------------------------------------------
\section{Acceptance correction}

\> as in NPB

%----------------------------------------------------------------------------------------------------
\section{Efficiency studies}

% OLD
\iffalse
\> \em{uncorrelated single-RP inefficiency}: $U_i$
\>> repeat elastic selection with 3 RPs of a diagonal and check the 4th RP

\> \em{correlated near-far inefficiency}: $C = 2\cdot (1.5 \pm 0.5)\un{\%}$
\>> study of events with characteristic signatures
\>> supported by MC simulation

\> \em{pile-up inefficiency}: $P$
\>> from zero-bias data (same bunches as for analysis)
\>> probability to find at least one (U or V) track pattern in both near and far RPs in any arm of a diagonal

\> formula for inefficiency correction per diagonal:\cBlack
$$\hbox{correction factor} = {1\over 1 - (\sum_{\rm RPs} U_i + C)} {1\over 1 - P}$$

\> trigger inefficiency
\>> zero-bias data, same bunches as in main analysis, determine:
\>>> number of events with elastic tag
\>>> number of events with elastic tag and RP trigger (trigger\_bits \& 1 == true)
\>> for all datasets, all diagonals, these two numbers come out equal
\>> source of zero-bias trigger
\>>> DS2, DS3: TOTEM, only on selected bunches
\>>> DS4: issued by CMS, on all 108 bunches (thus less events per bunch)
\>> lower bound for efficiency: $\ep_{min} = \root N \of {1 - CL}$
\>>> at CL = 95\%, $\ep > 99.5\un{\%}$
\>> details in \summary{efficiencies.ods}

\> DAQ inefficiency (= dead time)
\>> \plot{efficiencies/daq_efficiency.pdf} : DAQ efficiency
(= recorded / trigger events) per raw-data file
\>> included in effective luminosity: ${\cal L}^{int} = \sum\limits_{periods} {\cal L}_i^{int} \,\ep_i^{DAQ}$
\>> same mean DAQ efficiency if summing per run or per file
\>> \TODO: why DS4 has efficiency of $98\un{\%}$ only? (similar for all the 3 bunches)

\> summary spread-sheet: \summary{efficiencies.ods}

\subsection{3-out-of-4 efficiencies}

\> testing 1 RP at a time -- pot removed from tagging

\> tested arm -- including the test RP
\>> $\th_y^*$ determined from the other RP

\> reference arm -- opposite to the tested RP
\>> $\th_x^*$, $x^*$ and $\th_y^*$ determined as in standard analysis

\> cuts
\>> comparing arms: $\th_y^*$ cut
\>> reference arm:
\>>> $x^*$ compatible with vertex distribution
\>>> correlation $y^N$ vs. $\th_y$


\> plots:
\>> \plot{efficiencies/eff3outof4_details.pdf} : efficiency per RP as function of $\th_y^*$
\>> \plot{efficiencies/eff3outof4_2D.pdf} : efficiency per RP as function of $\th_x^*$ and $\th_y^*$
%\plot{efficiencies/eff3outof4_cuts.pdf}

\> traditional pattern: near pots efficiency about $98.5\un{\%}$, far about $97\un{\%}$

\> BUT: in all RPs, in all diagonal, in all datasets: the efficiency seems dropping with increasing $\th_y^*$
\>> the maximum difference is about $0.5\un{\%}$
\>> the effect stays no matter what cuts (only $\De^{R-L}\th_y^*$ cut or all applicable cuts) are applied
and no matter at what sigma level (1, 3 or 5)
\>> the efficiency decrease seems well compensated by increase of "enough planes but no track" ratio

\> the effect is well visible in 2D plots ($\th_x^*$ vs $\th_y^*$) too -- the decrease mainly follows the vertical direction

\> therefore, introduced $\th_y^*$ dependent efficiency correction
\>> the efficiency plateau is fitted with linear function for each RP: \plot{efficiencies/eff3outof4_details_fits.pdf}
\>> the total correction is given by the sum from the 4 diagonal RPs (eventually normalised such that its value at
$\th_y^* = 35\un{\mu rad}$ is zero -- to retain the overall normalisation)
\>> the uncertainty of the total slope correction is about $16\un{rad^{-1}}$ which corresponds to change in efficiency
of about $0.5\un{\%}$ between $30$ and $100\un{\mu rad}$

\> in \summary{efficiencies.ods}: results of "eff3outof4"
\>> $\th_y^*$ fits (flat plateau) at $3\un{\si}$; uncertainty to accommodate the slope of the efficiency plateau

\> for DS4, checked "uncorrelated 2-out-of-4": LF,RN; LF,RF; LN,RN; LN,RF -- perfect agreement with combining 3-out-of-4 results

\subsection{Pile-up}

\> pile-up probability vs.~time: \plot{efficiencies/pileup.pdf}

\> pile-up probability for individual bunches: \plot{efficiencies/pileup_per_bunch.pdf}
\>> DS4: similar results for all bunches, within $\pm 0.5\un{\%}$

\> standard procedure and expectable results: DS2 closer to beam, thus higher pileup ($\approx 7\un{\%}$) than DS3 and DS4
($\approx 1\un{\%}$).
\fi




%----------------------------------------------------------------------------------------------------
\section{Unfolding of resolution effects}

TODO

%----------------------------------------------------------------------------------------------------
\section{Normalisation}

\> as in NPB

%----------------------------------------------------------------------------------------------------
\section{Systematic uncertainties}



%----------------------------------------------------------------------------------------------------
\section{$t$ distributions}


\> \plot{t_distributions/t_dist.pdf}: comparison of datasets and diagonals





\bye
